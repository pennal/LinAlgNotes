\chapter{Linear Mappings}
\begin{definition}
A mapping $L:\R^m\to\R^n$ is said to be a linear mapping if for any $\uv,\vv\in\R^n$ and any scalar $\alpha$
\begin{align*}
L(\uv+\vv) &= L(\uv)+L(\vv)\\
L(\alpha\uv)&= \alpha L(\uv)
\end{align*}
for any matrix $A\in\R^{m,n}$ we can associate with it a linear mapping $L_A$ as
\[
\forall\uv\in\R^m\hspace{5mm}L_A(\uv)=A\uv
\]
In principle, any linear mapping is completely defined by its values on the basis vectors
\end{definition}
\begin{example}
Let us consider $L:\R^2\to\R^2$	and basis in $\R^2$
\[
\ul{b}_1 = \colvec{2}{1}{1},\ul{b}_2 = \colvec{2}{-1}{2}
\]
what will be $L\colvec{2}{1}{4}$?\\

Since $\ul{b}_1$ and $\ul{b}_2$ from basis in our space, $\colvec{2}{1}{4}$ can be represented as a linear combination of basis vectors
\[
\colvec{2}{1}{4} = \alpha_1\colvec{2}{1}{1} + \alpha_2\colvec{2}{-1}{2}
\]
we can find $\alpha_1$ and $\alpha_2$ ($\alpha_1=2$ and $\alpha_2=1$). Then 
\begin{align*}
L\colvec{2}{1}{4} &= L\left( 2\colvec{2}{1}{1}+1\colvec{2}{-1}{2} \right)\\
&= 2L\colvec{2}{1}{1}+L\colvec{2}{-1}{2}\\
&= 2\colvec{2}{7}{3}+\colvec{2}{10}{1}\\
&= \colvec{2}{14}{7}
\end{align*}
\end{example}

For any linear mapping, we can associate with it a matrix.
\begin{proof}
Consider the linear mapping $L:\R^m\to\R^n$. Consider also the standard basis
\[
E_1=\colvec{5}{1}{0}{0}{\vdots}{0},E_2=\colvec{5}{0}{1}{0}{\vdots}{0},\dots,E_n=\colvec{5}{0}{0}{0}{\vdots}{1}
\]
Let us denote by
\[
A_1=L(E_1),A_2=L(E_2),\dots,A_m=L(E_m)\in\R^n
\]
If we consider arbitrary vector $x\in\R^m$, then
\[
\ul{x}= = x_1E_1 + \dots + x_mE_m
\]
and also
\begin{align*}
L(\ul{x}) &= L(x_1E_1 + \dots + x_mE_m)\\
&= x_1L(E_1)+\dots+x_mL(E_m)\\
&= x_1A_1+\dots+x_mA_m\\
&= A\ul{x}
\end{align*}
where $A$ is a matrix whose columns are $A_1,A_2,\dots,A_m$. We found matrix $A$ associated with the linear mapping $L$. 
\end{proof}

\begin{example}
Consider linear mapping 
\[
L\colvec{3}{x_1}{x_2}{x_3} = \colvec{2}{x_1}{x_2} -\text{ projection from $\R^3$ to $\R^2$}
\]	
\end{example}

We consider

\begin{align*}
L(E_1) &= L\colvec{3}{1}{0}{0} = \colvec{2}{1}{0} = A_1\\
L(E_2) &= L\colvec{3}{0}{1}{0} = \colvec{2}{0}{1} = A_2\\
\end{align*}




