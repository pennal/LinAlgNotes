\chapter{Linear Mappings}
\begin{definition}
A mapping $L:\R^m\to\R^n$ is said to be a linear mapping if for any $\uv,\vv\in\R^n$ and any scalar $\alpha$
\begin{align*}
L(\uv+\vv) &= L(\uv)+L(\vv)\\
L(\alpha\uv)&= \alpha L(\uv)
\end{align*}
for any matrix $A\in\R^{m,n}$ we can associate with it a linear mapping $L_A$ as
\[
\forall\uv\in\R^m\hspace{5mm}L_A(\uv)=A\uv
\]
In principle, any linear mapping is completely defined by its values on the basis vectors
\end{definition}
\begin{example}
Let us consider $L:\R^2\to\R^2$	and basis in $\R^2$
\[
\ul{b}_1 = \colvec{2}{1}{1},\ul{b}_2 = \colvec{2}{-1}{2}
\]
what will be $L\colvec{2}{1}{4}$?\\

Since $\ul{b}_1$ and $\ul{b}_2$ from basis in our space, $\colvec{2}{1}{4}$ can be represented as a linear combination of basis vectors
\[
\colvec{2}{1}{4} = \alpha_1\colvec{2}{1}{1} + \alpha_2\colvec{2}{-1}{2}
\]
we can find $\alpha_1$ and $\alpha_2$ ($\alpha_1=2$ and $\alpha_2=1$). Then 
\begin{align*}
L\colvec{2}{1}{4} &= L\left( 2\colvec{2}{1}{1}+1\colvec{2}{-1}{2} \right)\\
&= 2L\colvec{2}{1}{1}+L\colvec{2}{-1}{2}\\
&= 2\colvec{2}{7}{3}+\colvec{2}{10}{1}\\
&= \colvec{2}{14}{7}
\end{align*}
\end{example}

For any linear mapping, we can associate with it a matrix.
\begin{proof}
Consider the linear mapping $L:\R^m\to\R^n$. Consider also the standard basis
\[
E_1=\colvec{5}{1}{0}{0}{\vdots}{0},E_2=\colvec{5}{0}{1}{0}{\vdots}{0},\dots,E_n=\colvec{5}{0}{0}{0}{\vdots}{1}
\]
Let us denote by
\[
A_1=L(E_1),A_2=L(E_2),\dots,A_m=L(E_m)\in\R^n
\]
If we consider arbitrary vector $x\in\R^m$, then
\[
\ul{x}= = x_1E_1 + \dots + x_mE_m
\]
and also
\begin{align*}
L(\ul{x}) &= L(x_1E_1 + \dots + x_mE_m)\\
&= x_1L(E_1)+\dots+x_mL(E_m)\\
&= x_1A_1+\dots+x_mA_m\\
&= A\ul{x}
\end{align*}
where $A$ is a matrix whose columns are $A_1,A_2,\dots,A_m$. We found matrix $A$ associated with the linear mapping $L$. 
\end{proof}

\begin{example}
Consider linear mapping 
\[
L\colvec{3}{x_1}{x_2}{x_3} = \colvec{2}{x_1}{x_2} -\text{ projection from $\R^3$ to $\R^2$}
\]	
\end{example}

We consider

\begin{align*}
L(E_1) &= L\colvec{3}{1}{0}{0} = \colvec{2}{1}{0} = A_1\\
L(E_2) &= L\colvec{3}{0}{1}{0} = \colvec{2}{0}{1} = A_2\\
L(E_3) &= L\colvec{3}{0}{0}{1} = \colvec{2}{0}{0} = A_3\\
\end{align*}
Then
\[
A = \begin{pmatrix}
1 & 0 & 0\\
0 & 1 & 0
\end{pmatrix}
\]
Matrix associated with linear mapping in a particular basis. From now on we will focus primarily on linear mappings $L:V\to V$. Assume $b_1,\dots,b_n$ form basis in space $V$ Then any vector $u\in V$ can be written as 
\[
\uv = u_1b_1+\dots +u_nb_n
\]
We can call $\colvec{3}{u_1}{\vdots}{u_n}\in\R^n$, the coordinates of $\uv$ in basis $b_1,\dots,b_n$.\\

Consider linear mapping $L:V\to V$. How does the matrix associated with $L$ look for basis $b_1,\dots,b_n$? Since $b_1,\dots,b_n$ is a basis of $V$, then we can write 
\begin{align*}
L(b_1\in V) &= c_{11}b_1+c_{12}b_2+\dots+c_{1n}b_n\\
&\hspace{2mm}\vdots\\
L(b_n\in V) &= c_{n1}b_1+c_{n2}b_2+\dots+c_{nn}b_n\\
\end{align*}
Now if we take an arbitrary vector $\uv\in V$
\[
\uv = u_1b_1+u_2b_2+\dots+u_nb_n = \sum\limits^{n}_{i=1}u_ib_i
\]
then
\begin{align*}
L(\uv) &= L\left( \sum\limits^{n}_{i=1}u_ib_i\right) = \sum\limits^{n}_{i=1}u_i L(b_i) = \sum\limits^{n}_{i=1}u_i\sum\limits^{n}_{j=1}c_{ij}b_j\\
&= \sum\limits^{n}_{i=1}\sum\limits^{n}_{j=1}u_ic_{ij}b_j = \sum\limits^{n}_{j=1}b_j\sum\limits^{n}_{i=1}u_ic_{ij}\\
&= \sum\limits^{n}_{i=1}c_{i1}u_i\times b_1 + \sum\limits^{n}_{i=1}c_{i2}u_i\times b_2 + \dots + \sum\limits^{n}_{i=1}c_{in}u_i\times b_n
\end{align*}
Therefore, we get
\[
L(\uv) = \colvec{3}{\sum\limits^{n}_{i=1}c_{i1}u_i}{\vdots}{\sum\limits^{n}_{i=1}c_{in}u_i} = C^Tu
\]
On coordinate vectors our linear mapping is represented by $L(\uv)=C^T\uv$ for a given basis $\ul{b}_1,\dots,\ul{b}_n$
\begin{note}
For a different basis we will have different coordinates of vectors as well as different associated matrix. 
\end{note}
\begin{example}
Consider $\R^3$ and basis $b_1,b_2,b_3$. Assume
\begin{align*}
L(b_1) &= b_1+b_2\\
L(b_2) &= 5b_1-b_2+3b_3\\
L(b_3) &= -b_1+4b_2-7b_3\\
\end{align*}
The matrix associated with this linear mapping is
\[
\begin{pmatrix}
1 & 5 & -1\\
1 & -1 & 4\\
0 & 3 & -7
\end{pmatrix} = C^T
\]
Let us say we have a vector whose coordinated in basis $b_1$, $b_2$ and $b_3$ are $\colvec{3}{1}{0}{0}$
\[
L\colvec{3}{1}{0}{0} = \begin{pmatrix}
1 & 5 & -1\\
1 & -1 & 4\\
0 & 3 & -7
\end{pmatrix}\colvec{3}{1}{0}{0} = \colvec{3}{1}{1}{0} = 1\cdot b_1 + 1\cdot b_2+0\cdot b_3
\]
\end{example}

\section{Change of Basis}
Let us first look at how coordinates of vectors change when we change the basis. Assume we have a vector space $V$. Let us also assume we have basis of $V$, $b_1,b_2,\dots,b_n$ and another basis $d_1,d_2,\dots,d_n$.\\

 Consider $\vv\in V$. Let $\colvec{3}{u_1}{\vdots}{u_n}$ be the coordinates of vector $\vv$ with respect to basis $b_1,b_2,\dots,b_n$ and $\colvec{3}{w_1}{\vdots}{w_n}$ be the coordinates of $\vv$ with respect to basis $d_1,d_2,\dots,d_n$.

\begin{align*}
\uv &= \uv_1\ul{b}_1+\dots +\uv_n\ul{b}_n = (u_1,\dots,u_n)\cdot\colvec{3}{b_1}{\vdots}{b_n} = \colvec{3}{u_1}{\vdots}{u_n}^T\colvec{3}{b_1}{\vdots}{b_n}\\
\vv &= \wv_1\ul{d}_1+\dots +\wv_n\ul{d}_n = (w_1,\dots,w_n)\cdot\colvec{3}{d_1}{\vdots}{d_n} = \colvec{3}{w_1}{\vdots}{w_n}^T\colvec{3}{d_1}{\vdots}{d_n}\\
\end{align*}
Since $\ul{b}_1,\dots,\ul{b}_n$ is a basis we can express each vector in a new basis $\ul{d}_1,\dots,\ul{d}_n$ in terms of $\ul{b}_1,\dots,\ul{b}_n$
\begin{align*}
d_1 &= S_{11}\ul{b}_1+S_{12}\ul{b}_2+\dots +S_{1n}\ul{b}_n\\
&\hspace{2mm}\vdots \\
d_n &= S_{n1}\ul{b}_1+S_{n2}\ul{b}_2+\dots +S_{nn}\ul{b}_n\\
\colvec{3}{\ul{d}_1}{\vdots}{\ul{d}_n} &= S\colvec{3}{\ul{b}_1}{\vdots}{\ul{b}_n}\\
\colvec{3}{\uv_1}{\vdots}{\uv_1}^T\colvec{3}{\ul{b}_1}{\dots}{\ul{b}_n} &= \colvec{3}{w_1}{\vdots}{w_n}^T\colvec{3}{\ul{d}_1}{\vdots}{\ul{d}_n} = \colvec{3}{\ul{w}_1}{\vdots}{\ul{w}_n}^TS\colvec{3}{\ul{b}_1}{\vdots}{\ul{b}_n}\\
\colvec{3}{\ul{u}_1}{\dots}{\ul{u}_n}^T &= \colvec{3}{\ul{w}_1}{\dots}{\ul{w}_n}^TS \mathop\Rightarrow\limits^{\hspace{4mm}(AB)^T=B^TA^T} \underbrace{\colvec{3}{u_1}{\vdots}{u_n}}_{N1} = \underbrace{S^T}_{N2}\underbrace{\colvec{3}{w_1}{\vdots}{w_n}}_{N3}
\end{align*}
\begin{itemize}
\item $N1$: Coordinates of $v$ in old basis $b_1,\dots,b_n$
\item Matrix $S$ describes $d_1,\dots,d_n$ with respect to basis $b_1,\dots,b_n$
\item Coordinates of $\vv$ in new basis $d_1,\dots,d_n$
\end{itemize}

\begin{lemma}
$S^T$ is invertible (i.e. $(S^T)^{-1}$ exists). We expressed $\colvec{3}{u_1}{\vdots}{u_n}$ as $S^T\colvec{3}{w_1}{\vdots}{w_n}$. We could do the same procedure, but exchanging $b_1,\dots,b_n$ with $d_1,\dots,d_n$ and we would arrive to 
\[
\colvec{3}{w_1}{\vdots}{w_n} = R^T\colvec{3}{u_1}{\vdots}{u_n}
\]
Now we have
\[
\begin{rcases*}
\wv = R^T\uv = R^TS^Tw\Rightarrow R^TS^T=I\\
\uv = S^T\wv = S^TR^T\uv\Rightarrow S^TR^T=I
\end{rcases*}\text{By def. of inverse }R^T=(S^T)^{-1}
\]
it means that $\wv = (S^T)^{-1}\uv$
\end{lemma}
\todo[inline]{Might be a title?}
How matrices associated with linear mappings change when we change the basis\\

Consider the linear mapping $L:V\to V$. Assume that $L$ is represented by matrix $A$ in basis $b_1,\dots,b_n$ and by matrix $A'$ in basis $d_1,\dots,d_n$. Consider Vector $\vv\in V$. Then in basis $b_1,\dots,b_n$
\[
L(\uv) = A\colvec{3}{u_1}{\vdots}{u_n}
\]
In basis $d_1,\dots,d_n$
\[
L(\uv) = A'\colvec{3}{w_1}{\vdots}{w_n}
\]
\[
A\colvec{3}{u_1}{\vdots}{u_n} = S^TAA\colvec{3}{w_1}{\vdots}{w_n}\Rightarrow AS^T\colvec{3}{w_1}{\vdots}{w_n} = S^TA'\colvec{3}{w_1}{\vdots}{w_n}
\]
Since $\wv$ is an ordinary vector
\[
\Rightarrow AS^T = S^TA\Rightarrow \underbrace{A'}_{N1} = (S^T)^{-1}\underbrace{A}_{N2}S^T
\]
\begin{itemize}
	\item $N1$: Matrix in new basis $d_1,\dots,d_n$
\item $N2$: Matrix in old basis $b_1,\dots,b_n$
\end{itemize}
The matrix associated with linear mappings changes as $A'=(S^T)^{-1}AS^T$ when we change from basis $b_1,\dots,b_n$ to $d_1,\dots,d_n$
\[
\colvec{3}{d_1}{\vdots}{d_n} = S\colvec{3}{b_1}{\vdots}{b_n}
\]

\begin{definition}
Assume that $N\in\R^{n,n}$, $N^{-1}$ exists. $A'=N^{-1}AN$ is called similarity transfromation
\end{definition}
\begin{definition}
Matrices $A'$ and $A$ are called similar matrices, if $\exists N$ such that 
\[
A'=N^{-1}AN
\]
\end{definition}

\begin{example}
Assume that linear mapping $L$ is represented with matrix 
\[
A = \begin{pmatrix}
1 & 4\\
2 & 3
\end{pmatrix}
\]	
with respect to basis 
\[
b_1 = \colvec{2}{1}{0}, b_2 = \colvec{2}{0}{1}
\]
Consider the new basis 
\[
d_1 = \colvec{2}{1}{1}, d_2 = \colvec{2}{1}{-\frac{1}{2}}
\]
How is $L$ represented with respect to the new basis?
\[
\begin{cases}
d_1 = 1\cdot b_1 + 1\cdot b_2\\
d_2 = 1\cdot b_1 - \frac{1}{2}\cdot b_2
\end{cases}
\]
\begin{align*}
\Rightarrow S &= \begin{pmatrix}
1 & 1\\
1 & -\frac{1}{2}
\end{pmatrix} \Rightarrow S^T =\begin{pmatrix}
1 & 1\\
1 & -\frac{1}{2}
\end{pmatrix}\\
(S^T)^{-1} &= -\frac{2}{3}\begin{pmatrix}
-\frac{1}{2} & -1\\
-1 & 1
\end{pmatrix}\\
A' &=  (S^T)^{-1} A S^T = -\frac{2}{3}\begin{pmatrix}
-\frac{1}{2} & -1\\
-1 & 1
\end{pmatrix}\begin{pmatrix}
1 & 4\\
2 & 3
\end{pmatrix}\begin{pmatrix}
1 & 1\\
1 & -\frac{1}{2}
\end{pmatrix}\\
&= \begin{pmatrix}
5 & 0\\
0 & -1
\end{pmatrix}
\end{align*}
In the new basis, our linear mapping is represented in a very simple way.

\end{example}



