\chapter{Matrices}
Let us consider a linear combination of vectors
\[
x_1\cdot\colvec{3}{u_1}{\vdots}{u_n}+x_2\cdot\colvec{3}{v_1}{\vdots}{v_n}+x_3\cdot\colvec{3}{w_1}{\vdots}{w_n} = \colvec{3}{b_1}{\vdots}{b_n}
\]
This can be written using matrices in the following way:
\[
\begin{pmatrix}
u_1 & v_1 & w_1\\
\vdots & \vdots & \vdots\\
u_n & v_n & w_n
\end{pmatrix}\cdot \colvec{3}{x_1}{x_2}{x_3} =
\colvec{3}{b_1}{\vdots}{b_n}
\]
In matrix-vector multiplication, we take dot products of rows of matrices times the vector.
\begin{example}
\begin{enumerate}
\item \[
\begin{pmatrix}
1 & 0 & -1\\
3 & 1 & 2\\
1 & -1 & 5\\
\end{pmatrix}
\cdot
\begin{pmatrix}
1\\
0\\
1\\
\end{pmatrix}
 = 
\begin{pmatrix} 
 1 \cdot 1 + 0 \cdot 0 + (-1) \cdot 1\\ 
 3 \cdot 1 + 1 \cdot 0 + 2 \cdot 1\\ 
 1 \cdot 1 + (-1) \cdot 0 + 5 \cdot 1\\
\end{pmatrix} 
 = 
 \begin{pmatrix} 
 0\\ 
 5\\ 
 6\\
\end{pmatrix}
\]
\begin{notation}
Matrices are usually written with capital letters, i.e. $A,B,c,\dots$\\

$A$ is an $n$ by $m$ matrix, $A\in\R^{n,m}$	 if it has $n$ rows and $m$ columns.\\

The element of matrix $A$ located in row $i$ and column $j$ is written as $a_{ij}$ or $(A)_{ij}$.
\end{notation}
\item 
\begin{align*}
A &= \begin{pmatrix}
-1 & 2 & 3\\
0 & 1 & 0
\end{pmatrix}, \ul{x} = \colvec{3}{1}{1}{1}\\
A\cdot\ul{x} &=\begin{pmatrix}
-1 & 2 & 3\\
0 & 1 & 0\\
\end{pmatrix}
\cdot
\begin{pmatrix}
1\\
1\\
1\\
\end{pmatrix}
 \\ &= 
\begin{pmatrix} 
 (-1) \cdot 1 + 2 \cdot 1 + 3 \cdot 1\\ 
 0 \cdot 1 + 1 \cdot 1 + 0 \cdot 1\\
\end{pmatrix} 
 = 
 \begin{pmatrix} 
 4\\ 
 1\\
\end{pmatrix}
\end{align*}
For the product of matrix $A$ with vector $\ul{x}$ to exist, matrix $A$ should have the same number of columns as vector $\ul{x}$ components.
\end{enumerate}
\end{example}

\section{Matrix Operations}
\begin{definition}
Let us consider matrices $A\in\R^{n,m}$ and $B\in\R^{n,m}$ where $n=$ rows, $m=$ columns. Matrix $C\in\R^{n,m}$ is a sum of $A$ and $B, C=A+B$ if $C_{ij} = A_{ij}+B_{ij}$ for all $i=1,\dots,n$, $j=1,\dots,m$
\end{definition}
\begin{example}
\[
A = \begin{pmatrix}
1 & 2\\
3 & 4\\
5 & 5
\end{pmatrix}, B = \begin{pmatrix}
-1 & 0\\
0 & -1\\
-1 & 0
\end{pmatrix}, C = A+B = \begin{pmatrix}
0 & 2\\
3 & 3\\
4 & 6
\end{pmatrix}
\]	
\end{example}
\begin{definition}
A product of a scalar $\alpha$ and a matrix $A\in\R^{n,m}$ is defined as $\left( \alpha A\right)_{ij} = \alpha\cdot A_{ij}$, $\forall i=1,\dots,n; j=1,\dots,m$.
\end{definition}
\begin{example}
\[\alpha = 3,A = \begin{pmatrix}
0 & 0 & 1\\
2 & 3 & 5	
\end{pmatrix}\Rightarrow \alpha\cdot A = \begin{pmatrix}
0 & 0 & 3\\
6 & 9 & 15
\end{pmatrix}
\]	
\end{example}
\begin{properties}
\begin{itemize}
\item $A\in\R^{n,m}$ and $B\in\R^{n,m}$: $A+B = B+A$\\
\ul{Proof:} 
\[
\begin{cases}
\left( A+B \right)_{ij} = A_{ij}+B_{ij}\\
\left( B+A \right)_{ij} = B_{ij}+A_{ij}
\end{cases}
\]
\item $A,B,C\in\R^{n,m}$: $(A+B)+C = A+(B+C)$
\item $\alpha\cdot (A+B) = \alpha A+\alpha B$ for $\forall \alpha\in\R, A,B\in\R^{n,m}$
\end{itemize}
	
\end{properties}

\section{Matrix - Matrix multiplication}
% TOP OF PAGE 10
