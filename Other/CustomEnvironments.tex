\newcommand{\R}{\mathbb{R}} % set R with \R
\newcommand{\N}{\mathbb{N}} % set N with \N
\newcommand{\C}{\mathbb{C}} % set C with \C
\newcommand{\uv}{\underline{u}}
\newcommand{\vv}{\underline{v}}
\newcommand{\wv}{\underline{w}}


\newcommand{\ul}[1]{\underline{#1}} % set C with \C

\newenvironment{definition}{\begin{framed}\centerline{\textbf{Definition}}\noindent\hspace{-1.1mm}}{\end{framed}}
\newenvironment{notation}{\subsubsection*{Notation}}{\begin{flushright}$\square$\end{flushright}}

\newenvironment{example}{\subsubsection*{Example}}{}%\begin{flushright}$\square$\end{flushright}}


% Source: http://tex.stackexchange.com/questions/2705/typesetting-column-vector. use with \colvec{5}{a}{b}{c}{d}{e}
\newcount\colveccount
\newcommand*\colvec[1]{
        \global\colveccount#1
        \begin{pmatrix}
        \colvecnext
}
\def\colvecnext#1{
        #1
        \global\advance\colveccount-1
        \ifnum\colveccount>0
                \\
                \expandafter\colvecnext
        \else
                \end{pmatrix}
        \fi
}